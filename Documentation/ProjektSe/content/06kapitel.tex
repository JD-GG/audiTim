\chapter{API-Kommunikation}
In diesem Kapitel wird die Kommunikation zwischen Frontend und Backend beschrieben, insbesondere die Übertragung von Heatmap-Daten. Es werden verschiedene Ansätze zur Datenübertragung und -kompression vorgestellt, um die Effizienz und Performance der Anwendung zu optimieren.
\section{Datenmengen und Optimierungsansätze}

\subsection{Problemstellung}
Wenn für jede Slider-Position eine API-Abfrage durchgeführt wird:
\begin{itemize}
    \item Bei einer Frequenz von 10~Hz (alle 100\,ms ein Datenpaket) ergeben sich 36{,}000 Zeitpunkte pro Sensor und somit 144{,}000 Werte pro Stunde.
    \item Beispielrechnung: $36{,}000 \times 10 \times 10 = 3{,}6$~Millionen Werte für ein $10\times10$-Array.
\end{itemize}

\subsection{Datenbasis}
\begin{itemize}
    \item \textbf{Heatmap-Größe:} $10\times10$ Werte (100 Felder pro Heatmap)
    \item \textbf{Datentyp:} Float16 (2~Byte pro Wert)
    \item \textbf{Frequenz:} 10~Hz (alle 100\,ms ein Frame)
    \item \textbf{Beispiel:} 1 Stunde = 3{,}600 Sekunden = 36{,}000 Frames
\end{itemize}

\noindent\textbf{Speicherbedarf (unkomprimiert):}
\[
100 \cdot 2\ \text{Bytes} = 200\ \text{Bytes/Frame}
\]
\[
200 \cdot 36{,}000 = 7{,}200{,}000\ \text{Bytes} \approx 7.2\ \text{MB}
\]

\subsection{Optimierungsoptionen}
\begin{enumerate}
    \item \textbf{Downsampling (Mittelwert pro Sekunde)} \\
    3{,}600 Frames/h → $3{,}600 \times 100$ Werte = 0.72\,MB.  
    Vorteil: geringe Datenmenge, einfaches JSON; Nachteil: Peaks nicht sichtbar.
    \item \textbf{Textkompression (Gzip/Brotli)} \\
    Reduktion auf ca. 15–25\,\% → $\approx 1.8$\,MB.
    \item \textbf{Binärformate (MessagePack, Protobuf, FlatBuffers)} \\
    30–40\,\% Ersparnis gegenüber JSON → $\approx 5.0$\,MB.
    \item \textbf{Binär + Kompression} \\
    Kombination aus beidem → $\approx 1.2$\,MB.
    \item \textbf{Delta-Encoding (nur Änderungen)} \\
    Angenommen: jeder 3.~Wert ändert sich → $\approx 3.4$\,MB.
    \item \textbf{Delta-Encoding + Kompression} \\
    Weitere Reduktion auf $\approx 1.0$–$1.2$\,MB.
    \item \textbf{Pre-Exportierte Dateien} \\
    Backend exportiert pro Stunde eine Datei → $\approx 7.2$\,MB.
    \item \textbf{Pre-Export + Kompression} \\
    Speicherbedarf: $\approx 1.2$\,MB/h; sehr performant und cachebar.
\end{enumerate}

\subsection{Vergleich der Ansätze}

\begin{center}
\begin{adjustbox}{max width=\textwidth}
\begin{tabular}{|l|c|p{5cm}|p{5cm}|}
\hline
\textbf{Beschreibung} & \textbf{Speicher/h} & \textbf{Vorteile} & \textbf{Nachteile} \\ \hline
Mittelwert pro Sekunde (Downsampling) & $\approx 0.72$\,MB & Sehr leicht, JSON möglich, wenig Traffic & Daten stark geglättet, Peaks nicht sichtbar \\ \hline
JSON + Gzip/Brotli & $\approx 1.8$\,MB & Einfach umsetzbar, keine großen Änderungen & Höhere Lesezeit, trotzdem noch Textformat \\ \hline
Binärformat & $\approx 5.0$\,MB & Schnell zu dekodieren, weniger Overhead & Komplexere Integration im Frontend \\ \hline
Binär + Kompression & $\approx 1.2$\,MB & Gute Mischung aus Effizienz \& Geschwindigkeit & Backend-Implementierung aufwändiger \\ \hline
Delta-Encoding & $\approx 3.5$\,MB & Ideal bei geringen Änderungen & Schlecht bei stark variablen Daten \\ \hline
Delta + Kompression & $1.0$–$1.2$\,MB & Sehr gute Kompression & Schwer zu debuggen, komplex \\ \hline
Pre-Export (roh) & $\approx 7.2$\,MB & Kein Runtime-Rechenaufwand, einfaches Caching & Hoher Speicherbedarf \\ \hline
Pre-Export + Kompression & $\approx 1.2$\,MB & Sehr performant, cachebar, ideal für Slider & Dateiverwaltung nötig \\ \hline
\end{tabular}
\end{adjustbox}
\end{center}

\subsection*{Fazit}
Aufgrund der hohen Komplexität und der großen Datenmengen ist es sinnvoll, die Heatmap-Arrays in einer separaten Datenbasis zu speichern. 
Das Frontend erhält die Daten in vorverarbeiteter Form durch Downsampling, um die Übertragungsmenge zu reduzieren und die Darstellungsgeschwindigkeit zu erhöhen. 
Zusätzlich wird eine Komprimierung der Daten in Betracht gezogen, um den Speicher- und Bandbreitenbedarf weiter zu minimieren, ohne die wesentliche Aussagekraft der Werte zu verlieren.
